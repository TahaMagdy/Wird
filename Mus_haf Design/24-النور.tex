%\documentclass[12pt,a4paper]{article}
\documentclass[20pt,a4paper]{article}
\usepackage[margin=0.5in]{geometry}
\usepackage{polyglossia}
\usepackage[dvipsnames]{xcolor}
\pagenumbering{gobble}
% This beautiful one line disable the initial spacing at the beginning of a line
\usepackage[parfill]{parskip} 
\usepackage{setspace}
\setstretch{2}

\setdefaultlanguage[numerals=maghrib]{arabic}
\newfontfamily\arabicfont[Script=Arabic]{Amiri}

\title{}
\author{}
\date{}
\definecolor{cl_page}{gray}{0.98}
\definecolor{cl_aya}{HTML}{DEEEFF}

\begin{document}
\pagecolor{cl_page}

% Start %


{\tiny\colorbox{cl_aya}{1}} سُورَةٌ أَنزَلْنَهَا وَفَرَضْنَهَا وَأَنزَلْنَا فِيهَا ءَايَتٍ بَيِّنَتٍ لَّعَلَّكُمْ تَذَكَّرُونَ
{\tiny\colorbox{cl_aya}{2}} الزَّانِيَةُ وَالزَّانِى فَاجْلِدُوا كُلَّ وَحِدٍ مِّنْهُمَا مِائَةَ جَلْدَةٍ وَلَا تَأْخُذْكُم بِهِمَا رَأْفَةٌ فِى دِينِ اللَّهِ إِن كُنتُمْ تُؤْمِنُونَ بِاللَّهِ وَالْيَوْمِ الْءَاخِرِ وَلْيَشْهَدْ عَذَابَهُمَا طَائِفَةٌ مِّنَ الْمُؤْمِنِينَ
{\tiny\colorbox{cl_aya}{3}} الزَّانِى لَا يَنكِحُ إِلَّا زَانِيَةً أَوْ مُشْرِكَةً وَالزَّانِيَةُ لَا يَنكِحُهَا إِلَّا زَانٍ أَوْ مُشْرِكٌ وَحُرِّمَ ذَلِكَ عَلَى الْمُؤْمِنِينَ
{\tiny\colorbox{cl_aya}{4}} وَالَّذِينَ يَرْمُونَ الْمُحْصَنَتِ ثُمَّ لَمْ يَأْتُوا بِأَرْبَعَةِ شُهَدَاءَ فَاجْلِدُوهُمْ ثَمَنِينَ جَلْدَةً وَلَا تَقْبَلُوا لَهُمْ شَهَدَةً أَبَدًا وَأُولَئِكَ هُمُ الْفَسِقُونَ
{\tiny\colorbox{cl_aya}{5}} إِلَّا الَّذِينَ تَابُوا مِن بَعْدِ ذَلِكَ وَأَصْلَحُوا فَإِنَّ اللَّهَ غَفُورٌ رَّحِيمٌ
{\tiny\colorbox{cl_aya}{6}} وَالَّذِينَ يَرْمُونَ أَزْوَجَهُمْ وَلَمْ يَكُن لَّهُمْ شُهَدَاءُ إِلَّا أَنفُسُهُمْ فَشَهَدَةُ أَحَدِهِمْ أَرْبَعُ شَهَدَتٍ بِاللَّهِ إِنَّهُ لَمِنَ الصَّدِقِينَ
{\tiny\colorbox{cl_aya}{7}} وَالْخَمِسَةُ أَنَّ لَعْنَتَ اللَّهِ عَلَيْهِ إِن كَانَ مِنَ الْكَذِبِينَ
{\tiny\colorbox{cl_aya}{8}} وَيَدْرَؤُا عَنْهَا الْعَذَابَ أَن تَشْهَدَ أَرْبَعَ شَهَدَتٍ بِاللَّهِ إِنَّهُ لَمِنَ الْكَذِبِينَ
{\tiny\colorbox{cl_aya}{9}} وَالْخَمِسَةَ أَنَّ غَضَبَ اللَّهِ عَلَيْهَا إِن كَانَ مِنَ الصَّدِقِينَ
{\tiny\colorbox{cl_aya}{10}} وَلَوْلَا فَضْلُ اللَّهِ عَلَيْكُمْ وَرَحْمَتُهُ وَأَنَّ اللَّهَ تَوَّابٌ حَكِيمٌ
{\tiny\colorbox{cl_aya}{11}} إِنَّ الَّذِينَ جَاءُو بِالْإِفْكِ عُصْبَةٌ مِّنكُمْ لَا تَحْسَبُوهُ شَرًّا لَّكُم بَلْ هُوَ خَيْرٌ لَّكُمْ لِكُلِّ امْرِئٍ مِّنْهُم مَّا اكْتَسَبَ مِنَ الْإِثْمِ وَالَّذِى تَوَلَّى كِبْرَهُ مِنْهُمْ لَهُ عَذَابٌ عَظِيمٌ
{\tiny\colorbox{cl_aya}{12}} لَّوْلَا إِذْ سَمِعْتُمُوهُ ظَنَّ الْمُؤْمِنُونَ وَالْمُؤْمِنَتُ بِأَنفُسِهِمْ خَيْرًا وَقَالُوا هَذَا إِفْكٌ مُّبِينٌ
{\tiny\colorbox{cl_aya}{13}} لَّوْلَا جَاءُو عَلَيْهِ بِأَرْبَعَةِ شُهَدَاءَ فَإِذْ لَمْ يَأْتُوا بِالشُّهَدَاءِ فَأُولَئِكَ عِندَ اللَّهِ هُمُ الْكَذِبُونَ
{\tiny\colorbox{cl_aya}{14}} وَلَوْلَا فَضْلُ اللَّهِ عَلَيْكُمْ وَرَحْمَتُهُ فِى الدُّنْيَا وَالْءَاخِرَةِ لَمَسَّكُمْ فِى مَا أَفَضْتُمْ فِيهِ عَذَابٌ عَظِيمٌ
{\tiny\colorbox{cl_aya}{15}} إِذْ تَلَقَّوْنَهُ بِأَلْسِنَتِكُمْ وَتَقُولُونَ بِأَفْوَاهِكُم مَّا لَيْسَ لَكُم بِهِ عِلْمٌ وَتَحْسَبُونَهُ هَيِّنًا وَهُوَ عِندَ اللَّهِ عَظِيمٌ
{\tiny\colorbox{cl_aya}{16}} وَلَوْلَا إِذْ سَمِعْتُمُوهُ قُلْتُم مَّا يَكُونُ لَنَا أَن نَّتَكَلَّمَ بِهَذَا سُبْحَنَكَ هَذَا بُهْتَنٌ عَظِيمٌ
{\tiny\colorbox{cl_aya}{17}} يَعِظُكُمُ اللَّهُ أَن تَعُودُوا لِمِثْلِهِ أَبَدًا إِن كُنتُم مُّؤْمِنِينَ
{\tiny\colorbox{cl_aya}{18}} وَيُبَيِّنُ اللَّهُ لَكُمُ الْءَايَتِ وَاللَّهُ عَلِيمٌ حَكِيمٌ
{\tiny\colorbox{cl_aya}{19}} إِنَّ الَّذِينَ يُحِبُّونَ أَن تَشِيعَ الْفَحِشَةُ فِى الَّذِينَ ءَامَنُوا لَهُمْ عَذَابٌ أَلِيمٌ فِى الدُّنْيَا وَالْءَاخِرَةِ وَاللَّهُ يَعْلَمُ وَأَنتُمْ لَا تَعْلَمُونَ
{\tiny\colorbox{cl_aya}{20}} وَلَوْلَا فَضْلُ اللَّهِ عَلَيْكُمْ وَرَحْمَتُهُ وَأَنَّ اللَّهَ رَءُوفٌ رَّحِيمٌ
{\tiny\colorbox{cl_aya}{21}} يَأَيُّهَا الَّذِينَ ءَامَنُوا لَا تَتَّبِعُوا خُطُوَتِ الشَّيْطَنِ وَمَن يَتَّبِعْ خُطُوَتِ الشَّيْطَنِ فَإِنَّهُ يَأْمُرُ بِالْفَحْشَاءِ وَالْمُنكَرِ وَلَوْلَا فَضْلُ اللَّهِ عَلَيْكُمْ وَرَحْمَتُهُ مَا زَكَى مِنكُم مِّنْ أَحَدٍ أَبَدًا وَلَكِنَّ اللَّهَ يُزَكِّى مَن يَشَاءُ وَاللَّهُ سَمِيعٌ عَلِيمٌ
{\tiny\colorbox{cl_aya}{22}} وَلَا يَأْتَلِ أُولُوا الْفَضْلِ مِنكُمْ وَالسَّعَةِ أَن يُؤْتُوا أُولِى الْقُرْبَى وَالْمَسَكِينَ وَالْمُهَجِرِينَ فِى سَبِيلِ اللَّهِ وَلْيَعْفُوا وَلْيَصْفَحُوا أَلَا تُحِبُّونَ أَن يَغْفِرَ اللَّهُ لَكُمْ وَاللَّهُ غَفُورٌ رَّحِيمٌ
{\tiny\colorbox{cl_aya}{23}} إِنَّ الَّذِينَ يَرْمُونَ الْمُحْصَنَتِ الْغَفِلَتِ الْمُؤْمِنَتِ لُعِنُوا فِى الدُّنْيَا وَالْءَاخِرَةِ وَلَهُمْ عَذَابٌ عَظِيمٌ
{\tiny\colorbox{cl_aya}{24}} يَوْمَ تَشْهَدُ عَلَيْهِمْ أَلْسِنَتُهُمْ وَأَيْدِيهِمْ وَأَرْجُلُهُم بِمَا كَانُوا يَعْمَلُونَ
{\tiny\colorbox{cl_aya}{25}} يَوْمَئِذٍ يُوَفِّيهِمُ اللَّهُ دِينَهُمُ الْحَقَّ وَيَعْلَمُونَ أَنَّ اللَّهَ هُوَ الْحَقُّ الْمُبِينُ
{\tiny\colorbox{cl_aya}{26}} الْخَبِيثَتُ لِلْخَبِيثِينَ وَالْخَبِيثُونَ لِلْخَبِيثَتِ وَالطَّيِّبَتُ لِلطَّيِّبِينَ وَالطَّيِّبُونَ لِلطَّيِّبَتِ أُولَئِكَ مُبَرَّءُونَ مِمَّا يَقُولُونَ لَهُم مَّغْفِرَةٌ وَرِزْقٌ كَرِيمٌ
{\tiny\colorbox{cl_aya}{27}} يَأَيُّهَا الَّذِينَ ءَامَنُوا لَا تَدْخُلُوا بُيُوتًا غَيْرَ بُيُوتِكُمْ حَتَّى تَسْتَأْنِسُوا وَتُسَلِّمُوا عَلَى أَهْلِهَا ذَلِكُمْ خَيْرٌ لَّكُمْ لَعَلَّكُمْ تَذَكَّرُونَ
{\tiny\colorbox{cl_aya}{28}} فَإِن لَّمْ تَجِدُوا فِيهَا أَحَدًا فَلَا تَدْخُلُوهَا حَتَّى يُؤْذَنَ لَكُمْ وَإِن قِيلَ لَكُمُ ارْجِعُوا فَارْجِعُوا هُوَ أَزْكَى لَكُمْ وَاللَّهُ بِمَا تَعْمَلُونَ عَلِيمٌ
{\tiny\colorbox{cl_aya}{29}} لَّيْسَ عَلَيْكُمْ جُنَاحٌ أَن تَدْخُلُوا بُيُوتًا غَيْرَ مَسْكُونَةٍ فِيهَا مَتَعٌ لَّكُمْ وَاللَّهُ يَعْلَمُ مَا تُبْدُونَ وَمَا تَكْتُمُونَ
{\tiny\colorbox{cl_aya}{30}} قُل لِّلْمُؤْمِنِينَ يَغُضُّوا مِنْ أَبْصَرِهِمْ وَيَحْفَظُوا فُرُوجَهُمْ ذَلِكَ أَزْكَى لَهُمْ إِنَّ اللَّهَ خَبِيرٌ بِمَا يَصْنَعُونَ
{\tiny\colorbox{cl_aya}{31}} وَقُل لِّلْمُؤْمِنَتِ يَغْضُضْنَ مِنْ أَبْصَرِهِنَّ وَيَحْفَظْنَ فُرُوجَهُنَّ وَلَا يُبْدِينَ زِينَتَهُنَّ إِلَّا مَا ظَهَرَ مِنْهَا وَلْيَضْرِبْنَ بِخُمُرِهِنَّ عَلَى جُيُوبِهِنَّ وَلَا يُبْدِينَ زِينَتَهُنَّ إِلَّا لِبُعُولَتِهِنَّ أَوْ ءَابَائِهِنَّ أَوْ ءَابَاءِ بُعُولَتِهِنَّ أَوْ أَبْنَائِهِنَّ أَوْ أَبْنَاءِ بُعُولَتِهِنَّ أَوْ إِخْوَنِهِنَّ أَوْ بَنِى إِخْوَنِهِنَّ أَوْ بَنِى أَخَوَتِهِنَّ أَوْ نِسَائِهِنَّ أَوْ مَا مَلَكَتْ أَيْمَنُهُنَّ أَوِ التَّبِعِينَ غَيْرِ أُولِى الْإِرْبَةِ مِنَ الرِّجَالِ أَوِ الطِّفْلِ الَّذِينَ لَمْ يَظْهَرُوا عَلَى عَوْرَتِ النِّسَاءِ وَلَا يَضْرِبْنَ بِأَرْجُلِهِنَّ لِيُعْلَمَ مَا يُخْفِينَ مِن زِينَتِهِنَّ وَتُوبُوا إِلَى اللَّهِ جَمِيعًا أَيُّهَ الْمُؤْمِنُونَ لَعَلَّكُمْ تُفْلِحُونَ
{\tiny\colorbox{cl_aya}{32}} وَأَنكِحُوا الْأَيَمَى مِنكُمْ وَالصَّلِحِينَ مِنْ عِبَادِكُمْ وَإِمَائِكُمْ إِن يَكُونُوا فُقَرَاءَ يُغْنِهِمُ اللَّهُ مِن فَضْلِهِ وَاللَّهُ وَسِعٌ عَلِيمٌ
{\tiny\colorbox{cl_aya}{33}} وَلْيَسْتَعْفِفِ الَّذِينَ لَا يَجِدُونَ نِكَاحًا حَتَّى يُغْنِيَهُمُ اللَّهُ مِن فَضْلِهِ وَالَّذِينَ يَبْتَغُونَ الْكِتَبَ مِمَّا مَلَكَتْ أَيْمَنُكُمْ فَكَاتِبُوهُمْ إِنْ عَلِمْتُمْ فِيهِمْ خَيْرًا وَءَاتُوهُم مِّن مَّالِ اللَّهِ الَّذِى ءَاتَىكُمْ وَلَا تُكْرِهُوا فَتَيَتِكُمْ عَلَى الْبِغَاءِ إِنْ أَرَدْنَ تَحَصُّنًا لِّتَبْتَغُوا عَرَضَ الْحَيَوةِ الدُّنْيَا وَمَن يُكْرِههُّنَّ فَإِنَّ اللَّهَ مِن بَعْدِ إِكْرَهِهِنَّ غَفُورٌ رَّحِيمٌ
{\tiny\colorbox{cl_aya}{34}} وَلَقَدْ أَنزَلْنَا إِلَيْكُمْ ءَايَتٍ مُّبَيِّنَتٍ وَمَثَلًا مِّنَ الَّذِينَ خَلَوْا مِن قَبْلِكُمْ وَمَوْعِظَةً لِّلْمُتَّقِينَ
{\tiny\colorbox{cl_aya}{35}} اللَّهُ نُورُ السَّمَوَتِ وَالْأَرْضِ مَثَلُ نُورِهِ كَمِشْكَوةٍ فِيهَا مِصْبَاحٌ الْمِصْبَاحُ فِى زُجَاجَةٍ الزُّجَاجَةُ كَأَنَّهَا كَوْكَبٌ دُرِّىٌّ يُوقَدُ مِن شَجَرَةٍ مُّبَرَكَةٍ زَيْتُونَةٍ لَّا شَرْقِيَّةٍ وَلَا غَرْبِيَّةٍ يَكَادُ زَيْتُهَا يُضِىءُ وَلَوْ لَمْ تَمْسَسْهُ نَارٌ نُّورٌ عَلَى نُورٍ يَهْدِى اللَّهُ لِنُورِهِ مَن يَشَاءُ وَيَضْرِبُ اللَّهُ الْأَمْثَلَ لِلنَّاسِ وَاللَّهُ بِكُلِّ شَىْءٍ عَلِيمٌ
{\tiny\colorbox{cl_aya}{36}} فِى بُيُوتٍ أَذِنَ اللَّهُ أَن تُرْفَعَ وَيُذْكَرَ فِيهَا اسْمُهُ يُسَبِّحُ لَهُ فِيهَا بِالْغُدُوِّ وَالْءَاصَالِ
{\tiny\colorbox{cl_aya}{37}} رِجَالٌ لَّا تُلْهِيهِمْ تِجَرَةٌ وَلَا بَيْعٌ عَن ذِكْرِ اللَّهِ وَإِقَامِ الصَّلَوةِ وَإِيتَاءِ الزَّكَوةِ يَخَافُونَ يَوْمًا تَتَقَلَّبُ فِيهِ الْقُلُوبُ وَالْأَبْصَرُ
{\tiny\colorbox{cl_aya}{38}} لِيَجْزِيَهُمُ اللَّهُ أَحْسَنَ مَا عَمِلُوا وَيَزِيدَهُم مِّن فَضْلِهِ وَاللَّهُ يَرْزُقُ مَن يَشَاءُ بِغَيْرِ حِسَابٍ
{\tiny\colorbox{cl_aya}{39}} وَالَّذِينَ كَفَرُوا أَعْمَلُهُمْ كَسَرَابٍ بِقِيعَةٍ يَحْسَبُهُ الظَّمَْٔانُ مَاءً حَتَّى إِذَا جَاءَهُ لَمْ يَجِدْهُ شَئًْا وَوَجَدَ اللَّهَ عِندَهُ فَوَفَّىهُ حِسَابَهُ وَاللَّهُ سَرِيعُ الْحِسَابِ
{\tiny\colorbox{cl_aya}{40}} أَوْ كَظُلُمَتٍ فِى بَحْرٍ لُّجِّىٍّ يَغْشَىهُ مَوْجٌ مِّن فَوْقِهِ مَوْجٌ مِّن فَوْقِهِ سَحَابٌ ظُلُمَتٌ بَعْضُهَا فَوْقَ بَعْضٍ إِذَا أَخْرَجَ يَدَهُ لَمْ يَكَدْ يَرَىهَا وَمَن لَّمْ يَجْعَلِ اللَّهُ لَهُ نُورًا فَمَا لَهُ مِن نُّورٍ
{\tiny\colorbox{cl_aya}{41}} أَلَمْ تَرَ أَنَّ اللَّهَ يُسَبِّحُ لَهُ مَن فِى السَّمَوَتِ وَالْأَرْضِ وَالطَّيْرُ صَفَّتٍ كُلٌّ قَدْ عَلِمَ صَلَاتَهُ وَتَسْبِيحَهُ وَاللَّهُ عَلِيمٌ بِمَا يَفْعَلُونَ
{\tiny\colorbox{cl_aya}{42}} وَلِلَّهِ مُلْكُ السَّمَوَتِ وَالْأَرْضِ وَإِلَى اللَّهِ الْمَصِيرُ
{\tiny\colorbox{cl_aya}{43}} أَلَمْ تَرَ أَنَّ اللَّهَ يُزْجِى سَحَابًا ثُمَّ يُؤَلِّفُ بَيْنَهُ ثُمَّ يَجْعَلُهُ رُكَامًا فَتَرَى الْوَدْقَ يَخْرُجُ مِنْ خِلَلِهِ وَيُنَزِّلُ مِنَ السَّمَاءِ مِن جِبَالٍ فِيهَا مِن بَرَدٍ فَيُصِيبُ بِهِ مَن يَشَاءُ وَيَصْرِفُهُ عَن مَّن يَشَاءُ يَكَادُ سَنَا بَرْقِهِ يَذْهَبُ بِالْأَبْصَرِ
{\tiny\colorbox{cl_aya}{44}} يُقَلِّبُ اللَّهُ الَّيْلَ وَالنَّهَارَ إِنَّ فِى ذَلِكَ لَعِبْرَةً لِّأُولِى الْأَبْصَرِ
{\tiny\colorbox{cl_aya}{45}} وَاللَّهُ خَلَقَ كُلَّ دَابَّةٍ مِّن مَّاءٍ فَمِنْهُم مَّن يَمْشِى عَلَى بَطْنِهِ وَمِنْهُم مَّن يَمْشِى عَلَى رِجْلَيْنِ وَمِنْهُم مَّن يَمْشِى عَلَى أَرْبَعٍ يَخْلُقُ اللَّهُ مَا يَشَاءُ إِنَّ اللَّهَ عَلَى كُلِّ شَىْءٍ قَدِيرٌ
{\tiny\colorbox{cl_aya}{46}} لَّقَدْ أَنزَلْنَا ءَايَتٍ مُّبَيِّنَتٍ وَاللَّهُ يَهْدِى مَن يَشَاءُ إِلَى صِرَطٍ مُّسْتَقِيمٍ
{\tiny\colorbox{cl_aya}{47}} وَيَقُولُونَ ءَامَنَّا بِاللَّهِ وَبِالرَّسُولِ وَأَطَعْنَا ثُمَّ يَتَوَلَّى فَرِيقٌ مِّنْهُم مِّن بَعْدِ ذَلِكَ وَمَا أُولَئِكَ بِالْمُؤْمِنِينَ
{\tiny\colorbox{cl_aya}{48}} وَإِذَا دُعُوا إِلَى اللَّهِ وَرَسُولِهِ لِيَحْكُمَ بَيْنَهُمْ إِذَا فَرِيقٌ مِّنْهُم مُّعْرِضُونَ
{\tiny\colorbox{cl_aya}{49}} وَإِن يَكُن لَّهُمُ الْحَقُّ يَأْتُوا إِلَيْهِ مُذْعِنِينَ
{\tiny\colorbox{cl_aya}{50}} أَفِى قُلُوبِهِم مَّرَضٌ أَمِ ارْتَابُوا أَمْ يَخَافُونَ أَن يَحِيفَ اللَّهُ عَلَيْهِمْ وَرَسُولُهُ بَلْ أُولَئِكَ هُمُ الظَّلِمُونَ
{\tiny\colorbox{cl_aya}{51}} إِنَّمَا كَانَ قَوْلَ الْمُؤْمِنِينَ إِذَا دُعُوا إِلَى اللَّهِ وَرَسُولِهِ لِيَحْكُمَ بَيْنَهُمْ أَن يَقُولُوا سَمِعْنَا وَأَطَعْنَا وَأُولَئِكَ هُمُ الْمُفْلِحُونَ
{\tiny\colorbox{cl_aya}{52}} وَمَن يُطِعِ اللَّهَ وَرَسُولَهُ وَيَخْشَ اللَّهَ وَيَتَّقْهِ فَأُولَئِكَ هُمُ الْفَائِزُونَ
{\tiny\colorbox{cl_aya}{53}} وَأَقْسَمُوا بِاللَّهِ جَهْدَ أَيْمَنِهِمْ لَئِنْ أَمَرْتَهُمْ لَيَخْرُجُنَّ قُل لَّا تُقْسِمُوا طَاعَةٌ مَّعْرُوفَةٌ إِنَّ اللَّهَ خَبِيرٌ بِمَا تَعْمَلُونَ
{\tiny\colorbox{cl_aya}{54}} قُلْ أَطِيعُوا اللَّهَ وَأَطِيعُوا الرَّسُولَ فَإِن تَوَلَّوْا فَإِنَّمَا عَلَيْهِ مَا حُمِّلَ وَعَلَيْكُم مَّا حُمِّلْتُمْ وَإِن تُطِيعُوهُ تَهْتَدُوا وَمَا عَلَى الرَّسُولِ إِلَّا الْبَلَغُ الْمُبِينُ
{\tiny\colorbox{cl_aya}{55}} وَعَدَ اللَّهُ الَّذِينَ ءَامَنُوا مِنكُمْ وَعَمِلُوا الصَّلِحَتِ لَيَسْتَخْلِفَنَّهُمْ فِى الْأَرْضِ كَمَا اسْتَخْلَفَ الَّذِينَ مِن قَبْلِهِمْ وَلَيُمَكِّنَنَّ لَهُمْ دِينَهُمُ الَّذِى ارْتَضَى لَهُمْ وَلَيُبَدِّلَنَّهُم مِّن بَعْدِ خَوْفِهِمْ أَمْنًا يَعْبُدُونَنِى لَا يُشْرِكُونَ بِى شَئًْا وَمَن كَفَرَ بَعْدَ ذَلِكَ فَأُولَئِكَ هُمُ الْفَسِقُونَ
{\tiny\colorbox{cl_aya}{56}} وَأَقِيمُوا الصَّلَوةَ وَءَاتُوا الزَّكَوةَ وَأَطِيعُوا الرَّسُولَ لَعَلَّكُمْ تُرْحَمُونَ
{\tiny\colorbox{cl_aya}{57}} لَا تَحْسَبَنَّ الَّذِينَ كَفَرُوا مُعْجِزِينَ فِى الْأَرْضِ وَمَأْوَىهُمُ النَّارُ وَلَبِئْسَ الْمَصِيرُ
{\tiny\colorbox{cl_aya}{58}} يَأَيُّهَا الَّذِينَ ءَامَنُوا لِيَسْتَْٔذِنكُمُ الَّذِينَ مَلَكَتْ أَيْمَنُكُمْ وَالَّذِينَ لَمْ يَبْلُغُوا الْحُلُمَ مِنكُمْ ثَلَثَ مَرَّتٍ مِّن قَبْلِ صَلَوةِ الْفَجْرِ وَحِينَ تَضَعُونَ ثِيَابَكُم مِّنَ الظَّهِيرَةِ وَمِن بَعْدِ صَلَوةِ الْعِشَاءِ ثَلَثُ عَوْرَتٍ لَّكُمْ لَيْسَ عَلَيْكُمْ وَلَا عَلَيْهِمْ جُنَاحٌ بَعْدَهُنَّ طَوَّفُونَ عَلَيْكُم بَعْضُكُمْ عَلَى بَعْضٍ كَذَلِكَ يُبَيِّنُ اللَّهُ لَكُمُ الْءَايَتِ وَاللَّهُ عَلِيمٌ حَكِيمٌ
{\tiny\colorbox{cl_aya}{59}} وَإِذَا بَلَغَ الْأَطْفَلُ مِنكُمُ الْحُلُمَ فَلْيَسْتَْٔذِنُوا كَمَا اسْتَْٔذَنَ الَّذِينَ مِن قَبْلِهِمْ كَذَلِكَ يُبَيِّنُ اللَّهُ لَكُمْ ءَايَتِهِ وَاللَّهُ عَلِيمٌ حَكِيمٌ
{\tiny\colorbox{cl_aya}{60}} وَالْقَوَعِدُ مِنَ النِّسَاءِ الَّتِى لَا يَرْجُونَ نِكَاحًا فَلَيْسَ عَلَيْهِنَّ جُنَاحٌ أَن يَضَعْنَ ثِيَابَهُنَّ غَيْرَ مُتَبَرِّجَتٍ بِزِينَةٍ وَأَن يَسْتَعْفِفْنَ خَيْرٌ لَّهُنَّ وَاللَّهُ سَمِيعٌ عَلِيمٌ
{\tiny\colorbox{cl_aya}{61}} لَّيْسَ عَلَى الْأَعْمَى حَرَجٌ وَلَا عَلَى الْأَعْرَجِ حَرَجٌ وَلَا عَلَى الْمَرِيضِ حَرَجٌ وَلَا عَلَى أَنفُسِكُمْ أَن تَأْكُلُوا مِن بُيُوتِكُمْ أَوْ بُيُوتِ ءَابَائِكُمْ أَوْ بُيُوتِ أُمَّهَتِكُمْ أَوْ بُيُوتِ إِخْوَنِكُمْ أَوْ بُيُوتِ أَخَوَتِكُمْ أَوْ بُيُوتِ أَعْمَمِكُمْ أَوْ بُيُوتِ عَمَّتِكُمْ أَوْ بُيُوتِ أَخْوَلِكُمْ أَوْ بُيُوتِ خَلَتِكُمْ أَوْ مَا مَلَكْتُم مَّفَاتِحَهُ أَوْ صَدِيقِكُمْ لَيْسَ عَلَيْكُمْ جُنَاحٌ أَن تَأْكُلُوا جَمِيعًا أَوْ أَشْتَاتًا فَإِذَا دَخَلْتُم بُيُوتًا فَسَلِّمُوا عَلَى أَنفُسِكُمْ تَحِيَّةً مِّنْ عِندِ اللَّهِ مُبَرَكَةً طَيِّبَةً كَذَلِكَ يُبَيِّنُ اللَّهُ لَكُمُ الْءَايَتِ لَعَلَّكُمْ تَعْقِلُونَ
{\tiny\colorbox{cl_aya}{62}} إِنَّمَا الْمُؤْمِنُونَ الَّذِينَ ءَامَنُوا بِاللَّهِ وَرَسُولِهِ وَإِذَا كَانُوا مَعَهُ عَلَى أَمْرٍ جَامِعٍ لَّمْ يَذْهَبُوا حَتَّى يَسْتَْٔذِنُوهُ إِنَّ الَّذِينَ يَسْتَْٔذِنُونَكَ أُولَئِكَ الَّذِينَ يُؤْمِنُونَ بِاللَّهِ وَرَسُولِهِ فَإِذَا اسْتَْٔذَنُوكَ لِبَعْضِ شَأْنِهِمْ فَأْذَن لِّمَن شِئْتَ مِنْهُمْ وَاسْتَغْفِرْ لَهُمُ اللَّهَ إِنَّ اللَّهَ غَفُورٌ رَّحِيمٌ
{\tiny\colorbox{cl_aya}{63}} لَّا تَجْعَلُوا دُعَاءَ الرَّسُولِ بَيْنَكُمْ كَدُعَاءِ بَعْضِكُم بَعْضًا قَدْ يَعْلَمُ اللَّهُ الَّذِينَ يَتَسَلَّلُونَ مِنكُمْ لِوَاذًا فَلْيَحْذَرِ الَّذِينَ يُخَالِفُونَ عَنْ أَمْرِهِ أَن تُصِيبَهُمْ فِتْنَةٌ أَوْ يُصِيبَهُمْ عَذَابٌ أَلِيمٌ
{\tiny\colorbox{cl_aya}{64}} أَلَا إِنَّ لِلَّهِ مَا فِى السَّمَوَتِ وَالْأَرْضِ قَدْ يَعْلَمُ مَا أَنتُمْ عَلَيْهِ وَيَوْمَ يُرْجَعُونَ إِلَيْهِ فَيُنَبِّئُهُم بِمَا عَمِلُوا وَاللَّهُ بِكُلِّ شَىْءٍ عَلِيمٌ
\end{document}