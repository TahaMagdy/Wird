%\documentclass[12pt,a4paper]{article}
\documentclass[20pt,a4paper]{article}
\usepackage[margin=0.5in]{geometry}
\usepackage{polyglossia}
\usepackage[dvipsnames]{xcolor}
\pagenumbering{gobble}
% This beautiful one line disable the initial spacing at the beginning of a line
\usepackage[parfill]{parskip} 
\usepackage{setspace}
\setstretch{2}

\setdefaultlanguage[numerals=maghrib]{arabic}
\newfontfamily\arabicfont[Script=Arabic]{Amiri}

\title{}
\author{}
\date{}
\definecolor{cl_page}{gray}{0.98}
\definecolor{cl_aya}{HTML}{DEEEFF}

\begin{document}
\pagecolor{cl_page}

% Start %


{\tiny\colorbox{cl_aya}{1}} طس تِلْكَ ءَايَتُ الْقُرْءَانِ وَكِتَابٍ مُّبِينٍ
{\tiny\colorbox{cl_aya}{2}} هُدًى وَبُشْرَى لِلْمُؤْمِنِينَ
{\tiny\colorbox{cl_aya}{3}} الَّذِينَ يُقِيمُونَ الصَّلَوةَ وَيُؤْتُونَ الزَّكَوةَ وَهُم بِالْءَاخِرَةِ هُمْ يُوقِنُونَ
{\tiny\colorbox{cl_aya}{4}} إِنَّ الَّذِينَ لَا يُؤْمِنُونَ بِالْءَاخِرَةِ زَيَّنَّا لَهُمْ أَعْمَلَهُمْ فَهُمْ يَعْمَهُونَ
{\tiny\colorbox{cl_aya}{5}} أُولَئِكَ الَّذِينَ لَهُمْ سُوءُ الْعَذَابِ وَهُمْ فِى الْءَاخِرَةِ هُمُ الْأَخْسَرُونَ
{\tiny\colorbox{cl_aya}{6}} وَإِنَّكَ لَتُلَقَّى الْقُرْءَانَ مِن لَّدُنْ حَكِيمٍ عَلِيمٍ
{\tiny\colorbox{cl_aya}{7}} إِذْ قَالَ مُوسَى لِأَهْلِهِ إِنِّى ءَانَسْتُ نَارًا سََٔاتِيكُم مِّنْهَا بِخَبَرٍ أَوْ ءَاتِيكُم بِشِهَابٍ قَبَسٍ لَّعَلَّكُمْ تَصْطَلُونَ
{\tiny\colorbox{cl_aya}{8}} فَلَمَّا جَاءَهَا نُودِىَ أَن بُورِكَ مَن فِى النَّارِ وَمَنْ حَوْلَهَا وَسُبْحَنَ اللَّهِ رَبِّ الْعَلَمِينَ
{\tiny\colorbox{cl_aya}{9}} يَمُوسَى إِنَّهُ أَنَا اللَّهُ الْعَزِيزُ الْحَكِيمُ
{\tiny\colorbox{cl_aya}{10}} وَأَلْقِ عَصَاكَ فَلَمَّا رَءَاهَا تَهْتَزُّ كَأَنَّهَا جَانٌّ وَلَّى مُدْبِرًا وَلَمْ يُعَقِّبْ يَمُوسَى لَا تَخَفْ إِنِّى لَا يَخَافُ لَدَىَّ الْمُرْسَلُونَ
{\tiny\colorbox{cl_aya}{11}} إِلَّا مَن ظَلَمَ ثُمَّ بَدَّلَ حُسْنًا بَعْدَ سُوءٍ فَإِنِّى غَفُورٌ رَّحِيمٌ
{\tiny\colorbox{cl_aya}{12}} وَأَدْخِلْ يَدَكَ فِى جَيْبِكَ تَخْرُجْ بَيْضَاءَ مِنْ غَيْرِ سُوءٍ فِى تِسْعِ ءَايَتٍ إِلَى فِرْعَوْنَ وَقَوْمِهِ إِنَّهُمْ كَانُوا قَوْمًا فَسِقِينَ
{\tiny\colorbox{cl_aya}{13}} فَلَمَّا جَاءَتْهُمْ ءَايَتُنَا مُبْصِرَةً قَالُوا هَذَا سِحْرٌ مُّبِينٌ
{\tiny\colorbox{cl_aya}{14}} وَجَحَدُوا بِهَا وَاسْتَيْقَنَتْهَا أَنفُسُهُمْ ظُلْمًا وَعُلُوًّا فَانظُرْ كَيْفَ كَانَ عَقِبَةُ الْمُفْسِدِينَ
{\tiny\colorbox{cl_aya}{15}} وَلَقَدْ ءَاتَيْنَا دَاوُدَ وَسُلَيْمَنَ عِلْمًا وَقَالَا الْحَمْدُ لِلَّهِ الَّذِى فَضَّلَنَا عَلَى كَثِيرٍ مِّنْ عِبَادِهِ الْمُؤْمِنِينَ
{\tiny\colorbox{cl_aya}{16}} وَوَرِثَ سُلَيْمَنُ دَاوُدَ وَقَالَ يَأَيُّهَا النَّاسُ عُلِّمْنَا مَنطِقَ الطَّيْرِ وَأُوتِينَا مِن كُلِّ شَىْءٍ إِنَّ هَذَا لَهُوَ الْفَضْلُ الْمُبِينُ
{\tiny\colorbox{cl_aya}{17}} وَحُشِرَ لِسُلَيْمَنَ جُنُودُهُ مِنَ الْجِنِّ وَالْإِنسِ وَالطَّيْرِ فَهُمْ يُوزَعُونَ
{\tiny\colorbox{cl_aya}{18}} حَتَّى إِذَا أَتَوْا عَلَى وَادِ النَّمْلِ قَالَتْ نَمْلَةٌ يَأَيُّهَا النَّمْلُ ادْخُلُوا مَسَكِنَكُمْ لَا يَحْطِمَنَّكُمْ سُلَيْمَنُ وَجُنُودُهُ وَهُمْ لَا يَشْعُرُونَ
{\tiny\colorbox{cl_aya}{19}} فَتَبَسَّمَ ضَاحِكًا مِّن قَوْلِهَا وَقَالَ رَبِّ أَوْزِعْنِى أَنْ أَشْكُرَ نِعْمَتَكَ الَّتِى أَنْعَمْتَ عَلَىَّ وَعَلَى وَلِدَىَّ وَأَنْ أَعْمَلَ صَلِحًا تَرْضَىهُ وَأَدْخِلْنِى بِرَحْمَتِكَ فِى عِبَادِكَ الصَّلِحِينَ
{\tiny\colorbox{cl_aya}{20}} وَتَفَقَّدَ الطَّيْرَ فَقَالَ مَا لِىَ لَا أَرَى الْهُدْهُدَ أَمْ كَانَ مِنَ الْغَائِبِينَ
{\tiny\colorbox{cl_aya}{21}} لَأُعَذِّبَنَّهُ عَذَابًا شَدِيدًا أَوْ لَأَاذْبَحَنَّهُ أَوْ لَيَأْتِيَنِّى بِسُلْطَنٍ مُّبِينٍ
{\tiny\colorbox{cl_aya}{22}} فَمَكَثَ غَيْرَ بَعِيدٍ فَقَالَ أَحَطتُ بِمَا لَمْ تُحِطْ بِهِ وَجِئْتُكَ مِن سَبَإٍ بِنَبَإٍ يَقِينٍ
{\tiny\colorbox{cl_aya}{23}} إِنِّى وَجَدتُّ امْرَأَةً تَمْلِكُهُمْ وَأُوتِيَتْ مِن كُلِّ شَىْءٍ وَلَهَا عَرْشٌ عَظِيمٌ
{\tiny\colorbox{cl_aya}{24}} وَجَدتُّهَا وَقَوْمَهَا يَسْجُدُونَ لِلشَّمْسِ مِن دُونِ اللَّهِ وَزَيَّنَ لَهُمُ الشَّيْطَنُ أَعْمَلَهُمْ فَصَدَّهُمْ عَنِ السَّبِيلِ فَهُمْ لَا يَهْتَدُونَ
{\tiny\colorbox{cl_aya}{25}} أَلَّا يَسْجُدُوا لِلَّهِ الَّذِى يُخْرِجُ الْخَبْءَ فِى السَّمَوَتِ وَالْأَرْضِ وَيَعْلَمُ مَا تُخْفُونَ وَمَا تُعْلِنُونَ
{\tiny\colorbox{cl_aya}{26}} اللَّهُ لَا إِلَهَ إِلَّا هُوَ رَبُّ الْعَرْشِ الْعَظِيمِ
{\tiny\colorbox{cl_aya}{27}} قَالَ سَنَنظُرُ أَصَدَقْتَ أَمْ كُنتَ مِنَ الْكَذِبِينَ
{\tiny\colorbox{cl_aya}{28}} اذْهَب بِّكِتَبِى هَذَا فَأَلْقِهْ إِلَيْهِمْ ثُمَّ تَوَلَّ عَنْهُمْ فَانظُرْ مَاذَا يَرْجِعُونَ
{\tiny\colorbox{cl_aya}{29}} قَالَتْ يَأَيُّهَا الْمَلَؤُا إِنِّى أُلْقِىَ إِلَىَّ كِتَبٌ كَرِيمٌ
{\tiny\colorbox{cl_aya}{30}} إِنَّهُ مِن سُلَيْمَنَ وَإِنَّهُ بِسْمِ اللَّهِ الرَّحْمَنِ الرَّحِيمِ
{\tiny\colorbox{cl_aya}{31}} أَلَّا تَعْلُوا عَلَىَّ وَأْتُونِى مُسْلِمِينَ
{\tiny\colorbox{cl_aya}{32}} قَالَتْ يَأَيُّهَا الْمَلَؤُا أَفْتُونِى فِى أَمْرِى مَا كُنتُ قَاطِعَةً أَمْرًا حَتَّى تَشْهَدُونِ
{\tiny\colorbox{cl_aya}{33}} قَالُوا نَحْنُ أُولُوا قُوَّةٍ وَأُولُوا بَأْسٍ شَدِيدٍ وَالْأَمْرُ إِلَيْكِ فَانظُرِى مَاذَا تَأْمُرِينَ
{\tiny\colorbox{cl_aya}{34}} قَالَتْ إِنَّ الْمُلُوكَ إِذَا دَخَلُوا قَرْيَةً أَفْسَدُوهَا وَجَعَلُوا أَعِزَّةَ أَهْلِهَا أَذِلَّةً وَكَذَلِكَ يَفْعَلُونَ
{\tiny\colorbox{cl_aya}{35}} وَإِنِّى مُرْسِلَةٌ إِلَيْهِم بِهَدِيَّةٍ فَنَاظِرَةٌ بِمَ يَرْجِعُ الْمُرْسَلُونَ
{\tiny\colorbox{cl_aya}{36}} فَلَمَّا جَاءَ سُلَيْمَنَ قَالَ أَتُمِدُّونَنِ بِمَالٍ فَمَا ءَاتَىنِ اللَّهُ خَيْرٌ مِّمَّا ءَاتَىكُم بَلْ أَنتُم بِهَدِيَّتِكُمْ تَفْرَحُونَ
{\tiny\colorbox{cl_aya}{37}} ارْجِعْ إِلَيْهِمْ فَلَنَأْتِيَنَّهُم بِجُنُودٍ لَّا قِبَلَ لَهُم بِهَا وَلَنُخْرِجَنَّهُم مِّنْهَا أَذِلَّةً وَهُمْ صَغِرُونَ
{\tiny\colorbox{cl_aya}{38}} قَالَ يَأَيُّهَا الْمَلَؤُا أَيُّكُمْ يَأْتِينِى بِعَرْشِهَا قَبْلَ أَن يَأْتُونِى مُسْلِمِينَ
{\tiny\colorbox{cl_aya}{39}} قَالَ عِفْرِيتٌ مِّنَ الْجِنِّ أَنَا ءَاتِيكَ بِهِ قَبْلَ أَن تَقُومَ مِن مَّقَامِكَ وَإِنِّى عَلَيْهِ لَقَوِىٌّ أَمِينٌ
{\tiny\colorbox{cl_aya}{40}} قَالَ الَّذِى عِندَهُ عِلْمٌ مِّنَ الْكِتَبِ أَنَا ءَاتِيكَ بِهِ قَبْلَ أَن يَرْتَدَّ إِلَيْكَ طَرْفُكَ فَلَمَّا رَءَاهُ مُسْتَقِرًّا عِندَهُ قَالَ هَذَا مِن فَضْلِ رَبِّى لِيَبْلُوَنِى ءَأَشْكُرُ أَمْ أَكْفُرُ وَمَن شَكَرَ فَإِنَّمَا يَشْكُرُ لِنَفْسِهِ وَمَن كَفَرَ فَإِنَّ رَبِّى غَنِىٌّ كَرِيمٌ
{\tiny\colorbox{cl_aya}{41}} قَالَ نَكِّرُوا لَهَا عَرْشَهَا نَنظُرْ أَتَهْتَدِى أَمْ تَكُونُ مِنَ الَّذِينَ لَا يَهْتَدُونَ
{\tiny\colorbox{cl_aya}{42}} فَلَمَّا جَاءَتْ قِيلَ أَهَكَذَا عَرْشُكِ قَالَتْ كَأَنَّهُ هُوَ وَأُوتِينَا الْعِلْمَ مِن قَبْلِهَا وَكُنَّا مُسْلِمِينَ
{\tiny\colorbox{cl_aya}{43}} وَصَدَّهَا مَا كَانَت تَّعْبُدُ مِن دُونِ اللَّهِ إِنَّهَا كَانَتْ مِن قَوْمٍ كَفِرِينَ
{\tiny\colorbox{cl_aya}{44}} قِيلَ لَهَا ادْخُلِى الصَّرْحَ فَلَمَّا رَأَتْهُ حَسِبَتْهُ لُجَّةً وَكَشَفَتْ عَن سَاقَيْهَا قَالَ إِنَّهُ صَرْحٌ مُّمَرَّدٌ مِّن قَوَارِيرَ قَالَتْ رَبِّ إِنِّى ظَلَمْتُ نَفْسِى وَأَسْلَمْتُ مَعَ سُلَيْمَنَ لِلَّهِ رَبِّ الْعَلَمِينَ
{\tiny\colorbox{cl_aya}{45}} وَلَقَدْ أَرْسَلْنَا إِلَى ثَمُودَ أَخَاهُمْ صَلِحًا أَنِ اعْبُدُوا اللَّهَ فَإِذَا هُمْ فَرِيقَانِ يَخْتَصِمُونَ
{\tiny\colorbox{cl_aya}{46}} قَالَ يَقَوْمِ لِمَ تَسْتَعْجِلُونَ بِالسَّيِّئَةِ قَبْلَ الْحَسَنَةِ لَوْلَا تَسْتَغْفِرُونَ اللَّهَ لَعَلَّكُمْ تُرْحَمُونَ
{\tiny\colorbox{cl_aya}{47}} قَالُوا اطَّيَّرْنَا بِكَ وَبِمَن مَّعَكَ قَالَ طَئِرُكُمْ عِندَ اللَّهِ بَلْ أَنتُمْ قَوْمٌ تُفْتَنُونَ
{\tiny\colorbox{cl_aya}{48}} وَكَانَ فِى الْمَدِينَةِ تِسْعَةُ رَهْطٍ يُفْسِدُونَ فِى الْأَرْضِ وَلَا يُصْلِحُونَ
{\tiny\colorbox{cl_aya}{49}} قَالُوا تَقَاسَمُوا بِاللَّهِ لَنُبَيِّتَنَّهُ وَأَهْلَهُ ثُمَّ لَنَقُولَنَّ لِوَلِيِّهِ مَا شَهِدْنَا مَهْلِكَ أَهْلِهِ وَإِنَّا لَصَدِقُونَ
{\tiny\colorbox{cl_aya}{50}} وَمَكَرُوا مَكْرًا وَمَكَرْنَا مَكْرًا وَهُمْ لَا يَشْعُرُونَ
{\tiny\colorbox{cl_aya}{51}} فَانظُرْ كَيْفَ كَانَ عَقِبَةُ مَكْرِهِمْ أَنَّا دَمَّرْنَهُمْ وَقَوْمَهُمْ أَجْمَعِينَ
{\tiny\colorbox{cl_aya}{52}} فَتِلْكَ بُيُوتُهُمْ خَاوِيَةً بِمَا ظَلَمُوا إِنَّ فِى ذَلِكَ لَءَايَةً لِّقَوْمٍ يَعْلَمُونَ
{\tiny\colorbox{cl_aya}{53}} وَأَنجَيْنَا الَّذِينَ ءَامَنُوا وَكَانُوا يَتَّقُونَ
{\tiny\colorbox{cl_aya}{54}} وَلُوطًا إِذْ قَالَ لِقَوْمِهِ أَتَأْتُونَ الْفَحِشَةَ وَأَنتُمْ تُبْصِرُونَ
{\tiny\colorbox{cl_aya}{55}} أَئِنَّكُمْ لَتَأْتُونَ الرِّجَالَ شَهْوَةً مِّن دُونِ النِّسَاءِ بَلْ أَنتُمْ قَوْمٌ تَجْهَلُونَ
{\tiny\colorbox{cl_aya}{56}} فَمَا كَانَ جَوَابَ قَوْمِهِ إِلَّا أَن قَالُوا أَخْرِجُوا ءَالَ لُوطٍ مِّن قَرْيَتِكُمْ إِنَّهُمْ أُنَاسٌ يَتَطَهَّرُونَ
{\tiny\colorbox{cl_aya}{57}} فَأَنجَيْنَهُ وَأَهْلَهُ إِلَّا امْرَأَتَهُ قَدَّرْنَهَا مِنَ الْغَبِرِينَ
{\tiny\colorbox{cl_aya}{58}} وَأَمْطَرْنَا عَلَيْهِم مَّطَرًا فَسَاءَ مَطَرُ الْمُنذَرِينَ
{\tiny\colorbox{cl_aya}{59}} قُلِ الْحَمْدُ لِلَّهِ وَسَلَمٌ عَلَى عِبَادِهِ الَّذِينَ اصْطَفَى ءَاللَّهُ خَيْرٌ أَمَّا يُشْرِكُونَ
{\tiny\colorbox{cl_aya}{60}} أَمَّنْ خَلَقَ السَّمَوَتِ وَالْأَرْضَ وَأَنزَلَ لَكُم مِّنَ السَّمَاءِ مَاءً فَأَنبَتْنَا بِهِ حَدَائِقَ ذَاتَ بَهْجَةٍ مَّا كَانَ لَكُمْ أَن تُنبِتُوا شَجَرَهَا أَءِلَهٌ مَّعَ اللَّهِ بَلْ هُمْ قَوْمٌ يَعْدِلُونَ
{\tiny\colorbox{cl_aya}{61}} أَمَّن جَعَلَ الْأَرْضَ قَرَارًا وَجَعَلَ خِلَلَهَا أَنْهَرًا وَجَعَلَ لَهَا رَوَسِىَ وَجَعَلَ بَيْنَ الْبَحْرَيْنِ حَاجِزًا أَءِلَهٌ مَّعَ اللَّهِ بَلْ أَكْثَرُهُمْ لَا يَعْلَمُونَ
{\tiny\colorbox{cl_aya}{62}} أَمَّن يُجِيبُ الْمُضْطَرَّ إِذَا دَعَاهُ وَيَكْشِفُ السُّوءَ وَيَجْعَلُكُمْ خُلَفَاءَ الْأَرْضِ أَءِلَهٌ مَّعَ اللَّهِ قَلِيلًا مَّا تَذَكَّرُونَ
{\tiny\colorbox{cl_aya}{63}} أَمَّن يَهْدِيكُمْ فِى ظُلُمَتِ الْبَرِّ وَالْبَحْرِ وَمَن يُرْسِلُ الرِّيَحَ بُشْرًا بَيْنَ يَدَىْ رَحْمَتِهِ أَءِلَهٌ مَّعَ اللَّهِ تَعَلَى اللَّهُ عَمَّا يُشْرِكُونَ
{\tiny\colorbox{cl_aya}{64}} أَمَّن يَبْدَؤُا الْخَلْقَ ثُمَّ يُعِيدُهُ وَمَن يَرْزُقُكُم مِّنَ السَّمَاءِ وَالْأَرْضِ أَءِلَهٌ مَّعَ اللَّهِ قُلْ هَاتُوا بُرْهَنَكُمْ إِن كُنتُمْ صَدِقِينَ
{\tiny\colorbox{cl_aya}{65}} قُل لَّا يَعْلَمُ مَن فِى السَّمَوَتِ وَالْأَرْضِ الْغَيْبَ إِلَّا اللَّهُ وَمَا يَشْعُرُونَ أَيَّانَ يُبْعَثُونَ
{\tiny\colorbox{cl_aya}{66}} بَلِ ادَّرَكَ عِلْمُهُمْ فِى الْءَاخِرَةِ بَلْ هُمْ فِى شَكٍّ مِّنْهَا بَلْ هُم مِّنْهَا عَمُونَ
{\tiny\colorbox{cl_aya}{67}} وَقَالَ الَّذِينَ كَفَرُوا أَءِذَا كُنَّا تُرَبًا وَءَابَاؤُنَا أَئِنَّا لَمُخْرَجُونَ
{\tiny\colorbox{cl_aya}{68}} لَقَدْ وُعِدْنَا هَذَا نَحْنُ وَءَابَاؤُنَا مِن قَبْلُ إِنْ هَذَا إِلَّا أَسَطِيرُ الْأَوَّلِينَ
{\tiny\colorbox{cl_aya}{69}} قُلْ سِيرُوا فِى الْأَرْضِ فَانظُرُوا كَيْفَ كَانَ عَقِبَةُ الْمُجْرِمِينَ
{\tiny\colorbox{cl_aya}{70}} وَلَا تَحْزَنْ عَلَيْهِمْ وَلَا تَكُن فِى ضَيْقٍ مِّمَّا يَمْكُرُونَ
{\tiny\colorbox{cl_aya}{71}} وَيَقُولُونَ مَتَى هَذَا الْوَعْدُ إِن كُنتُمْ صَدِقِينَ
{\tiny\colorbox{cl_aya}{72}} قُلْ عَسَى أَن يَكُونَ رَدِفَ لَكُم بَعْضُ الَّذِى تَسْتَعْجِلُونَ
{\tiny\colorbox{cl_aya}{73}} وَإِنَّ رَبَّكَ لَذُو فَضْلٍ عَلَى النَّاسِ وَلَكِنَّ أَكْثَرَهُمْ لَا يَشْكُرُونَ
{\tiny\colorbox{cl_aya}{74}} وَإِنَّ رَبَّكَ لَيَعْلَمُ مَا تُكِنُّ صُدُورُهُمْ وَمَا يُعْلِنُونَ
{\tiny\colorbox{cl_aya}{75}} وَمَا مِنْ غَائِبَةٍ فِى السَّمَاءِ وَالْأَرْضِ إِلَّا فِى كِتَبٍ مُّبِينٍ
{\tiny\colorbox{cl_aya}{76}} إِنَّ هَذَا الْقُرْءَانَ يَقُصُّ عَلَى بَنِى إِسْرَءِيلَ أَكْثَرَ الَّذِى هُمْ فِيهِ يَخْتَلِفُونَ
{\tiny\colorbox{cl_aya}{77}} وَإِنَّهُ لَهُدًى وَرَحْمَةٌ لِّلْمُؤْمِنِينَ
{\tiny\colorbox{cl_aya}{78}} إِنَّ رَبَّكَ يَقْضِى بَيْنَهُم بِحُكْمِهِ وَهُوَ الْعَزِيزُ الْعَلِيمُ
{\tiny\colorbox{cl_aya}{79}} فَتَوَكَّلْ عَلَى اللَّهِ إِنَّكَ عَلَى الْحَقِّ الْمُبِينِ
{\tiny\colorbox{cl_aya}{80}} إِنَّكَ لَا تُسْمِعُ الْمَوْتَى وَلَا تُسْمِعُ الصُّمَّ الدُّعَاءَ إِذَا وَلَّوْا مُدْبِرِينَ
{\tiny\colorbox{cl_aya}{81}} وَمَا أَنتَ بِهَدِى الْعُمْىِ عَن ضَلَلَتِهِمْ إِن تُسْمِعُ إِلَّا مَن يُؤْمِنُ بَِٔايَتِنَا فَهُم مُّسْلِمُونَ
{\tiny\colorbox{cl_aya}{82}} وَإِذَا وَقَعَ الْقَوْلُ عَلَيْهِمْ أَخْرَجْنَا لَهُمْ دَابَّةً مِّنَ الْأَرْضِ تُكَلِّمُهُمْ أَنَّ النَّاسَ كَانُوا بَِٔايَتِنَا لَا يُوقِنُونَ
{\tiny\colorbox{cl_aya}{83}} وَيَوْمَ نَحْشُرُ مِن كُلِّ أُمَّةٍ فَوْجًا مِّمَّن يُكَذِّبُ بَِٔايَتِنَا فَهُمْ يُوزَعُونَ
{\tiny\colorbox{cl_aya}{84}} حَتَّى إِذَا جَاءُو قَالَ أَكَذَّبْتُم بَِٔايَتِى وَلَمْ تُحِيطُوا بِهَا عِلْمًا أَمَّاذَا كُنتُمْ تَعْمَلُونَ
{\tiny\colorbox{cl_aya}{85}} وَوَقَعَ الْقَوْلُ عَلَيْهِم بِمَا ظَلَمُوا فَهُمْ لَا يَنطِقُونَ
{\tiny\colorbox{cl_aya}{86}} أَلَمْ يَرَوْا أَنَّا جَعَلْنَا الَّيْلَ لِيَسْكُنُوا فِيهِ وَالنَّهَارَ مُبْصِرًا إِنَّ فِى ذَلِكَ لَءَايَتٍ لِّقَوْمٍ يُؤْمِنُونَ
{\tiny\colorbox{cl_aya}{87}} وَيَوْمَ يُنفَخُ فِى الصُّورِ فَفَزِعَ مَن فِى السَّمَوَتِ وَمَن فِى الْأَرْضِ إِلَّا مَن شَاءَ اللَّهُ وَكُلٌّ أَتَوْهُ دَخِرِينَ
{\tiny\colorbox{cl_aya}{88}} وَتَرَى الْجِبَالَ تَحْسَبُهَا جَامِدَةً وَهِىَ تَمُرُّ مَرَّ السَّحَابِ صُنْعَ اللَّهِ الَّذِى أَتْقَنَ كُلَّ شَىْءٍ إِنَّهُ خَبِيرٌ بِمَا تَفْعَلُونَ
{\tiny\colorbox{cl_aya}{89}} مَن جَاءَ بِالْحَسَنَةِ فَلَهُ خَيْرٌ مِّنْهَا وَهُم مِّن فَزَعٍ يَوْمَئِذٍ ءَامِنُونَ
{\tiny\colorbox{cl_aya}{90}} وَمَن جَاءَ بِالسَّيِّئَةِ فَكُبَّتْ وُجُوهُهُمْ فِى النَّارِ هَلْ تُجْزَوْنَ إِلَّا مَا كُنتُمْ تَعْمَلُونَ
{\tiny\colorbox{cl_aya}{91}} إِنَّمَا أُمِرْتُ أَنْ أَعْبُدَ رَبَّ هَذِهِ الْبَلْدَةِ الَّذِى حَرَّمَهَا وَلَهُ كُلُّ شَىْءٍ وَأُمِرْتُ أَنْ أَكُونَ مِنَ الْمُسْلِمِينَ
{\tiny\colorbox{cl_aya}{92}} وَأَنْ أَتْلُوَا الْقُرْءَانَ فَمَنِ اهْتَدَى فَإِنَّمَا يَهْتَدِى لِنَفْسِهِ وَمَن ضَلَّ فَقُلْ إِنَّمَا أَنَا مِنَ الْمُنذِرِينَ
{\tiny\colorbox{cl_aya}{93}} وَقُلِ الْحَمْدُ لِلَّهِ سَيُرِيكُمْ ءَايَتِهِ فَتَعْرِفُونَهَا وَمَا رَبُّكَ بِغَفِلٍ عَمَّا تَعْمَلُونَ
\end{document}